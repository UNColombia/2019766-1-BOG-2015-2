
\documentclass{llncs}

\usepackage{makeidx}  
\usepackage[spanish,colombia]{babel}
\begin{document}
\frontmatter          
\pagestyle{headings}  
\addtocmark{Optimización matemática}
\mainmatter              
\title{Optimización matemática}
\titlerunning{Optimización matemática}
\author{Jefferson Javier Hernandez Panqueba \and Jorge Iván Andrés Contreras Pereira}
\institute{Universidad Nacional de Colombia, Bogotá, Colombia,\\
\email{\{jejhernandezpa,jacontrerasp\}@unal.edu.co}
}

\maketitle

\begin{abstract}
The abstract should summarize the contents of the paper
using at least 70 and at most 150 words. It will be set in 9-point
font size and be inset 1.0 cm from the right and left margins.
There will be two blank lines before and after the Abstract. 
\keywords{computación evolutiva, optimización,optimización matemática, método de Newton-Raphson}
\end{abstract}
%
\section{Introducción}

\section{Marco teórico}
\begin{document}
Sea f : $\text{\textgreek{W}}\text{\ensuremath{\subseteq}}R^{n}\rightarrow R$ una función. La hipersuperficie de nivel, superficie de nivel si n
= 3, curva de nivel si n = 2, está definida por \{S = x \ensuremath{\in}
\textgreek{W} \textbar{} f(x) = c\}, constante real

Ya que la derivada direccional de f (x) en la dirección de d = \ensuremath{\nabla}f(x)
/ \textbar{}\textbar{} \ensuremath{\nabla}f(x)\textbar{}\textbar{}
, \textbar{}\textbar{}d\textbar{}\textbar{} = 1, está dada por:
$\frac{\text{\ensuremath{\partial}}f(x)}{\text{\ensuremath{\partial}}d}=\text{\textlangle\ensuremath{\nabla}f(x),d\textrangle}=||\text{\ensuremath{\nabla}}f(x)|| ||d|| cos \theta =||\text{\ensuremath{\nabla}}f(x)||$,

entonces :
\begin{itemize}
\item La función f (x) crece más rápidamente en la dirección de\ensuremath{\nabla}f(x). 
\item La función f (x) decrece más rápidamente en la dirección de -\ensuremath{\nabla}f(x).
\item Cualquier dirección $u\text{\ensuremath{\in}}R^{n}$ ortogonal al
\ensuremath{\nabla}f(x) es una dirección de cambio nulo.
\end{itemize}
Dirección de búsqueda

La dirección negativa del gradiente, -\ensuremath{\nabla}f(x), es una buena dirección de búsqueda para encontrar un minimizador de la función.




\begin{figure}
\vspace{2.5cm}
\caption{This is the caption of the figure displaying a white eagle and
a white horse on a snow field}
\end{figure}

\begin{equation}
  \dot{x} = JH' (t,x)
\end{equation}
such that, for every $k\in \bbbn$, there is some $p_{o}\in\bbbn$ with:
\begin{equation}
  p\ge p_{o}\Rightarrow x_{pk} \ne x_{k}\ .
\end{equation}
\qed
\end{theorem}
%
\begin{example} [{{\rm External forcing}}]
Consider the system:
\begin{equation}
  \dot{x} = JH' (x) + f(t)
\end{equation}
where the Hamiltonian $H$ is
$\left(0,b_{\infty}\right)$-subquadratic, and the
forcing term is a distribution on the circle:
\begin{equation}
  f = \frac{d}{dt} F + f_{o}\ \ \ \ \
  {\rm with}\ \ F\in L^{2} \left(\bbbr / T\bbbz; \bbbr^{2n}\right)\ ,
\end{equation}
where $f_{o} : = T^{-1}\int_{o}^{T} f (t) dt$. For instance,
\begin{equation}
  f (t) = \sum_{k\in \bbbn} \delta_{k} \xi\ ,
\end{equation}
where $\delta_{k}$ is the Dirac mass at $t= k$ and
$\xi \in \bbbr^{2n}$ is a
constant, fits the prescription. This means that the system
$\dot{x} = JH' (x)$ is being excited by a
series of identical shocks at interval $T$.
\end{example}
%
\begin{definition}
Let $A_{\infty} (t)$ and $B_{\infty} (t)$ be symmetric
operators in $\bbbr^{2n}$, depending continuously on
$t\in [0,T]$, such that
$A_{\infty} (t) \le B_{\infty} (t)$ for all $t$.

A Borelian function
$H: [0,T]\times \bbbr^{2n} \to \bbbr$
is called
$\left(A_{\infty} ,B_{\infty}\right)$-{\it subquadratic at infinity}
if there exists a function $N(t,x)$ such that:
\begin{equation}
  H (t,x) = \frac{1}{2} \left(A_{\infty} (t) x,x\right) + N(t,x)
\end{equation}
\begin{equation}
  \forall t\ ,\ \ \ N(t,x)\ \ \ \ \
  {\rm is\ convex\ with\  respect\  to}\ \ x
\end{equation}
\begin{equation}
  N(t,x) \ge n\left(\left\|x\right\|\right)\ \ \ \ \
  {\rm with}\ \ n(s)s^{-1}\to +\infty\ \ {\rm as}\ \ s\to +\infty
\end{equation}
\begin{equation}
  \exists c\in \bbbr\ :\ \ \ H (t,x) \le
  \frac{1}{2} \left(B_{\infty} (t) x,x\right) + c\ \ \ \forall x\ .
\end{equation}

If $A_{\infty} (t) = a_{\infty} I$ and
$B_{\infty} (t) = b_{\infty} I$, with
$a_{\infty} \le b_{\infty} \in \bbbr$,
we shall say that $H$ is
$\left(a_{\infty},b_{\infty}\right)$-subquadratic
at infinity. As an example, the function
$\left\|x\right\|^{\alpha}$, with
$1\le \alpha < 2$, is $(0,\varepsilon )$-subquadratic at infinity
for every $\varepsilon > 0$. Similarly, the Hamiltonian
\begin{equation}
H (t,x) = \frac{1}{2} k \left\|k\right\|^{2} +\left\|x\right\|^{\alpha}
\end{equation}
is $(k,k+\varepsilon )$-subquadratic for every $\varepsilon > 0$.
Note that, if $k<0$, it is not convex.
\end{definition}
%

\paragraph{Notes and Comments.}
The first results on subharmonics were
obtained by Rabinowitz in \cite{2rab}, who showed the existence of
infinitely many subharmonics both in the subquadratic and superquadratic
case, with suitable growth conditions on $H'$. Again the duality
approach enabled Clarke and Ekeland in \cite{2clar:eke:2} to treat the
same problem in the convex-subquadratic case, with growth conditions on
$H$ only.

Recently, Michalek and Tarantello (see Michalek, R., Tarantello, G.
\cite{2mich:tar} and Tarantello, G. \cite{2tar}) have obtained lower
bound on the number of subharmonics of period $kT$, based on symmetry
considerations and on pinching estimates, as in Sect.~5.2 of this
article.

%
% ---- Bibliography ----
%
\begin{thebibliography}{}
%
\bibitem[1980]{2clar:eke}
Clarke, F., Ekeland, I.:
Nonlinear oscillations and
boundary-value problems for Hamiltonian systems.
Arch. Rat. Mech. Anal. 78, 315--333 (1982)

\bibitem[1981]{2clar:eke:2}
Clarke, F., Ekeland, I.:
Solutions p\'{e}riodiques, du
p\'{e}riode donn\'{e}e, des \'{e}quations hamiltoniennes.
Note CRAS Paris 287, 1013--1015 (1978)

\bibitem[1982]{2mich:tar}
Michalek, R., Tarantello, G.:
Subharmonic solutions with prescribed minimal
period for nonautonomous Hamiltonian systems.
J. Diff. Eq. 72, 28--55 (1988)

\bibitem[1983]{2tar}
Tarantello, G.:
Subharmonic solutions for Hamiltonian
systems via a $\bbbz_{p}$ pseudoindex theory.
Annali di Matematica Pura (to appear)

\bibitem[1985]{2rab}
Rabinowitz, P.:
On subharmonic solutions of a Hamiltonian system.
Comm. Pure Appl. Math. 33, 609--633 (1980)

\end{thebibliography}
\clearpage
\addtocmark[2]{Author Index} % additional numbered TOC entry
\renewcommand{\indexname}{Author Index}
\printindex
\clearpage
\end{document}
